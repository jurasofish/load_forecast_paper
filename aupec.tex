\documentclass[conference]{IEEEtran}
\IEEEoverridecommandlockouts
% The preceding line is only needed to identify funding in the first footnote. If that is unneeded, please comment it out.
\usepackage{cite}
\usepackage{amsmath,amssymb,amsfonts}
\usepackage{algorithmic}
\usepackage{graphicx}
\usepackage{textcomp}
\usepackage{xcolor}
\def\BibTeX{{\rm B\kern-.05em{\sc i\kern-.025em b}\kern-.08em
    T\kern-.1667em\lower.7ex\hbox{E}\kern-.125emX}}
\begin{document}

\title{Day Ahead Load Forecasting for the Modern Distribution Network - A Case Study in the Tasmanian Distribution Network}

\author{\IEEEauthorblockN{Michael Jurasovic}
\IEEEauthorblockA{\textit{Engineering~~} \\
\textit{UTAS~~}\\
Hobart~ \\
mjj4@utas.edu.au}
\and
\IEEEauthorblockN{Evan Franklin}
\IEEEauthorblockA{\textit{dept. name of organization (of Aff.)} \\
\textit{name of organization (of Aff.)}\\
City, Country \\
email address}
\and
\IEEEauthorblockN{Michael Negnevitsky}
\IEEEauthorblockA{\textit{dept. name of organization (of Aff.)} \\
\textit{name of organization (of Aff.)}\\
City, Country \\
email address}
}

\maketitle

\begin{abstract}
The transformer neural network architecture was applied to short term load forecasting.
\end{abstract}

\begin{IEEEkeywords}
component, formatting, style, styling, insert
\end{IEEEkeywords}

\section{Introduction}
The modern distribution network has changed more over the last ten years than it has in the previous hundred.
In the past, generation and load were largely separate; power was generated exclusively at large stations, and power was consumed by customers after traversing the transmission and distribution networks. 
These days, power is still consumed in the distribution network, but is also generated and manipulated by distributed energy resources (DER). 
\par
DERs are controllable devices in the power network that generate, store, and consume load. 
This includes solar generation (PV), battery storage, and electric vehicles (EV). 
\par
The Tasmanian distribution network is forecast to experience significant increases in these technologies by 2025: \\
\begin{itemize}
	\item 600\% increase in battery storage capacity (from 100MWh to 600MWh) \cite{Jacobs2017}
	\item 170\% increase in PV installation capacity (from 130MW to 220MW) \cite{Jacobs2017}
	\item 39\% of new car sales with be EVs - the highest in the country \cite{AEMO2016}
\end{itemize}

This changing network presents an opportunity to maximize the use of existing assets by delaying the need for network augmentations, while also providing customers with a more reliable supply of power.
For example, batteries could be used to peak-shift, reducing maximum feeder load.
However, to achieve this requires sophisticated methods to optimize the power flow to and from the distributed resources.
\par
One method to achieve this is presented in \cite{Scott2014} and has been implemented on Bruny Island, Tasmania.
The island is a popular holiday destination and during peak periods, such as Easter morning/afternoon peaks, the submarine feeder supplying the island becomes overloaded and has to be supplemented by a diesel generator located on the island.
The aim of the project was to peak shift the load away from the morning/afternoon and avoid the use of the generator.

The method relies on having an accurate forecast of day-ahead load at the feeder level.
Load forecasting methods commonly employed in industry are neither intended to forecast with high accuracy over a time period this short nor at such a low level in the distribution network \cite{CIGRE2016}.
As a result, the optimization of the distributed resources was not as effective as it could be.
\par
To solve this, a neural network-based load forecasting system is proposed.
This system will be applied to Bruny Island, in southern Tasmania, as a case study.
Bruny Island currently has a high penetration of PV and battery technology which is optimized by the Network-aware Coordination (NAC) algorithm \cite{Evan2016}.
Additionally, it is constrained by its feeder during peak holiday periods, necessitating an on-island diesel generator. The load forecasting system will need to be able to perform equally well on holidays, where the load is generally large, and on normal days. 
This makes it a perfect case study to highlight the potential benefits that distributed resources can have in the network.
This project and case study is supported by TasNetworks.


\section{Proposed Forecasting System}
Recurrent Neural Networks (RNN) have recently been popular for load forecasting \cite{Kong2018}.
However, RNNs have been out-performed by the Transformer model in several domains including machine translation \cite{Vaswani2017}, medical time series forecasting and regression \cite{Song2017}, and image generation \cite{Parmar2018}.
The proposed forecasting system uses a Transformer neural network model combined with similar day selection.
\par
Given a multivariate time series $X = (x_1, ..., x_T)$, with $x_t \in \mathbb{R}^N$ denoting a point in time comprised of N observations, the forecasting system takes X and produces a univariate time series forecast $Y = (y_1, ..., y_U)$ with $y_t \in \mathbb{R}^1$.

\subsection{Similar Day}
Simple Euclidean distance between max/min temp, day of week, holiday type.

\subsection{Transformer}
Use Attention is all You Need, and Attend and Diagnose as reference.
Sequence in, sequence out. Self attention.

\section{Case Study}
The forecasting system was applied to Bruny Island.
Also discuss application to a common dataset for comparison with other papers?

\subsection{Data}
Discussion of available data and what was supplied to the forecasting system.

\subsection{Results}
Results of case study.

\section{Conclusion}
Blah Blah the forecaster works.
Maybe something about future work?


\section*{Acknowledgment}
TNW


\bibliographystyle{IEEEtran}
\bibliography{aupec}

\end{document}
