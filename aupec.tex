\documentclass[conference]{IEEEtran}
\IEEEoverridecommandlockouts
% The preceding line is only needed to identify funding in the first footnote. If that is unneeded, please comment it out.
\usepackage{cite}
\usepackage{amsmath,amssymb,amsfonts}
\usepackage{algorithmic}
\usepackage{graphicx}
\usepackage{textcomp}
\usepackage{xcolor}
\def\BibTeX{{\rm B\kern-.05em{\sc i\kern-.025em b}\kern-.08em
    T\kern-.1667em\lower.7ex\hbox{E}\kern-.125emX}}
\begin{document}

\title{Day Ahead Load Forecasting for the Modern Distribution Network}

\author{\IEEEauthorblockN{Michael Jurasovic}
\IEEEauthorblockA{\textit{Engineering~~} \\
\textit{UTAS~~}\\
Hobart~ \\
mjj4@utas.edu.au}
\and
\IEEEauthorblockN{Evan Franklin}
\IEEEauthorblockA{\textit{dept. name of organization (of Aff.)} \\
\textit{name of organization (of Aff.)}\\
City, Country \\
email address}
\and
\IEEEauthorblockN{Michael Negnevitsky}
\IEEEauthorblockA{\textit{dept. name of organization (of Aff.)} \\
\textit{name of organization (of Aff.)}\\
City, Country \\
email address}
}

\maketitle

\begin{abstract}
The transformer neural network architecture was applied to short term load forecasting.
\end{abstract}

\begin{IEEEkeywords}
component, formatting, style, styling, insert
\end{IEEEkeywords}

\section{Introduction}
Load forecasting is important blah blah, distributed energy resources, batteries, aggregate levels of load to be forecast (household, feeder, city), overview of method.
\cite{Vaswani2017} % need at least one citation to compile.

\section{Proposed Forecasting System}
Recurrent Neural Networks (RNN) have recently been popular for load forecasting (cite blah).
Transformer models have achieved greater performance than RNNs in many language tasks (openAI).
(Transformer now justified for research).
This forecasting system uses a Transformer neural network model combined with similar day selection.
\par
This paper will present the transformer model, and then discuss its application in a case study.

\subsection{Transformer}
Use Attention is all You Need, and Attend and Diagnose as reference.
Sequence in, sequence out. Self attention.

\subsection{Similar Day}
Simple Euclidean distance between max/min temp, day of week, holiday type.

\section{Case Study}
The forecasting system was applied to Bruny Island.
Also discuss application to a common dataset for comparison with other papers?

\subsection{Data}
Discussion of available data and what was supplied to the forecasting system.

\subsection{Results}
Results of case study.

\section{Conclusion}
Blah Blah the forecaster works.
Maybe something about future work?


\section*{Acknowledgment}
TNW


\bibliographystyle{IEEEtran}
\bibliography{aupec}

\end{document}
